%% start of file `template.tex'.
%% Copyright 2006-2013 Xavier Danaux (xdanaux@gmail.com).
%
% This work may be distributed and/or modified under the
% conditions of the LaTeX Project Public License version 1.3c,
% available at http://www.latex-project.org/lppl/.


\documentclass[11pt,a4paper,sans]{moderncv}        % possible options include font size ('10pt', '11pt' and '12pt'), paper size ('a4paper', 'letterpaper', 'a5paper', 'legalpaper', 'executivepaper' and 'landscape') and font family ('sans' and 'roman')

% moderncv themes
\moderncvstyle{casual}                             % style options are 'casual' (default), 'classic', 'oldstyle' and 'banking'
\moderncvcolor{blue}                               % color options 'blue' (default), 'orange', 'green', 'red', 'purple', 'grey' and 'black'
\setmainfont[Ligatures={Common, TeX}]{Linux Libertine O}
%\renewcommand{\familydefault}{\sfdefault}         % to set the default font; use '\sfdefault' for the default sans serif font, '\rmdefault' for the default roman one, or any tex font name
%\nopagenumbers{}                                  % uncomment to suppress automatic page numbering for CVs longer than one page

\usepackage{xpatch}
\xpatchcmd{\makeletterclosing}{[3em]}{[0em]}{}{}

% justified text
\usepackage{etoolbox}
\makeatletter
\patchcmd{\makeletterhead}
    {\raggedright \@opening}
    {\@opening}
    {}{}
\makeatother
% adjust the page margins
\usepackage[scale=0.75]{geometry}
%\setlength{\hintscolumnwidth}{3cm}                % if you want to change the width of the column with the dates

% personal data
\name{Jared}{Garst}

%----------------------------------------------------------------------------------
%            content
%----------------------------------------------------------------------------------
\begin{document}
%-----       letter       ---------------------------------------------------------
% recipient data
\recipient{~}{~}
\date{\today}
\opening{Dear Human Dx,}
\closing{All the best,}
\makelettertitle%
I'm pleased to offer my resume for the Software Engineer role at Human Dx.
I have spent the last three years helping people collect, store, chop up and understand their data.
Continuing this work by providing infrastructure for open medical data would be a privilege.


There are a few reasons I could be a good Human Dx engineer --- experience working with researchers, familiarity with natural language processing, and practice with data analysis.
I'd like to call out my teaching skills specifically though.
They are under rated as part of good engineering practice, and I find it strange that more people aren't practicing them.
I teach the introductory python class Monday nights at Noisebridge, and have spent previous years as a physics, statistics and software educator.
The hardest problems at my work are tasks like ``determine and explain how the legacy code works'' or ``generate executive buy-in for a different software direction''.
Having communication experience gives me a starting point for these problems, and pedagogical techniques like ``Don't teach facts, teach models'' help me evaluate and refine the results.


I am also proud of the technical work I have done, and hope you will be impressed with the bullet points on my resume.
But this is the sort of work that seems the most important, the most fraught, and which most keeps me up at night.
I think it is an excellent reason to hire me.


\makeletterclosing%

\end{document}


%% end of file `template.tex'.
