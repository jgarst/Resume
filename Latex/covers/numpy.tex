%% start of file `template.tex'.
%% Copyright 2006-2013 Xavier Danaux (xdanaux@gmail.com).
%
% This work may be distributed and/or modified under the
% conditions of the LaTeX Project Public License version 1.3c,
% available at http://www.latex-project.org/lppl/.


\documentclass[11pt,a4paper,sans]{moderncv}        % possible options include font size ('10pt', '11pt' and '12pt'), paper size ('a4paper', 'letterpaper', 'a5paper', 'legalpaper', 'executivepaper' and 'landscape') and font family ('sans' and 'roman')

% moderncv themes
\moderncvstyle{casual}                             % style options are 'casual' (default), 'classic', 'oldstyle' and 'banking'
\moderncvcolor{black}                               % color options 'blue' (default), 'orange', 'green', 'red', 'purple', 'grey' and 'black'
\setmainfont[Ligatures={Common, TeX}]{Linux Libertine O}
%\renewcommand{\familydefault}{\sfdefault}         % to set the default font; use '\sfdefault' for the default sans serif font, '\rmdefault' for the default roman one, or any tex font name
%\nopagenumbers{}                                  % uncomment to suppress automatic page numbering for CVs longer than one page

% justified text
\usepackage{etoolbox}
\makeatletter
\patchcmd{\makeletterhead}
    {\raggedright \@opening}
    {\@opening}
    {}{}
\makeatother
% adjust the page margins
\usepackage[scale=0.75]{geometry}
%\setlength{\hintscolumnwidth}{3cm}                % if you want to change the width of the column with the dates

% personal data
\name{Jared}{Garst}

%----------------------------------------------------------------------------------
%            content
%----------------------------------------------------------------------------------
\begin{document}
%-----       letter       ---------------------------------------------------------
% recipient data
\recipient{~}{~}
\date{\today}
\opening{Dear BIDS team,}
\closing{All the best,}
\makelettertitle

I am delighted to submit a resume for the role of open source python developer.
As a user, proponent, and educator in the scientific python stack, I could not hope for more meaningful work.

The theme of my work so far has been analysis workflows that are reliable and learnable.
When I taught physicists how to do their first data analysis, I took notes on what was difficult for them.
I use those notes in my current work, writing scientific software for people who have better things to worry about.
In addition to thinking constantly of how build software without stumbling blocks, I spend lots of energy communicating with our customers on every available channel.
When customers complain about slow queries, I succinctly outline our plans for speed improvement and ways they can make their query faster.
I keep notes on our most confused bug reports, with the tools people used and the messages they saw.
When customers have feature suggestions I take those seriously and provide additional context.

I hope to use these skills, as well as my abilities in Python, in service of NumPy and its community.

\makeletterclosing

\end{document}


%% end of file `template.tex'.
